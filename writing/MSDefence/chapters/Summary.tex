\section{Conclusion}

\begin{frame}{Summary and Future directions}
  \begin{itemize}
    \item<1-> Resource levels can control strength of competition
    \item<2-> Balance of limitations promote coexistence
    \item<3-> With standard-of-care: testosterone limitation is increased leading to extinction
    \item<4-> With adaptive therapy: competitive release avoided
    \begin{itemize}
      \item Effectiveness depends on $T^+$ and $T^p$ population
      \item Population controlled by resource limitations
      \item Maximum limit on $T^+$ and $T^p$ by thresholds of adaptive therapy
    \end{itemize}
    \item<5-> Future directions:
    \begin{itemize}
      \item<6-> Make adaptive therapy effective at reducing tumour burden
      \item<7-> Dynamic thresholds for turning on/off based on composition of the tumour
      \item<8-> Different limitations for different cell types
    \end{itemize}
  \end{itemize}
\end{frame}

\begin{frame}{Acknowledgement}
  I would like to thank the following people:
  \begin{itemize}
    \item Supervisor: Prof. Sutirth Dey
    \item Expert: Dr. M.S. Madhusudhan
    \item Mentor: Vibishan B
    \item PBL Members
    \item Friends and Family
    \item KVPY and IISER Pune
  \end{itemize}
\end{frame}
