\chapter{Discussion}
The findings of the study lead to the following points of general interest, beginning with the pairwise interactions. For the $T^p - T^-$ pairwise competition, the limitation of testosterone for $T^p$ and that of oxygen for $T^-$ has a major influence on competition and its outcomes. Increasing the limitation of a given resource for one cell relative to the other leads to the more-limited cell going extinct, regardless of specific identity. Only when the limitations are balanced between two types was coexistence observed. Resource levels can therefore act as control levers of the strength of competitive interactions between cell types and therefore determine the feasibility of coexistence. In addition to resource limitations, the relative initial seeding proportion of the cells can push the outcome in favour of the dominant cell. $T^-$ in conventional terms has an advantage due to its shorter doubling time and the requirement of one less resource. A model that doesn’t account for explicit resource dynamics and limitations might therefore predict that $T^-$ always wins. Although competition coefficients can be made to disfavour $T^-$ , justification for such an artefact isn’t straightforward, particularly compared to the emergence of coexistence we observed in our model due to resource limitation effects.

Similar to $T^p - T^-$, the relative limitation for a given resource of one cell over the other, as well as the relative seeding proportion of the cells, both influence the outcomes of the $T^+ - T^p$ pairwise competition. Resource limitations can in fact be directly compared here unlike $T^p - T^-$, as both the cell types share the same qualitative resource dependencies. $T^p$ has an advantage over $T^+$ with oxygen limitations as the latter requires the former for testosterone. However, with testosterone limitation, $T^p$ has a disadvantage relative to $T^+$ as it would be doubly growth-limited by both testosterone and $T^+$ density. Even though symmetric limitation of a resource across both the cell types produces a similar effect, the difference with testosterone is much more pronounced than with oxygen limitation, possibly indicating a stronger dependence on testosterone for these two cell types.

The general trend discussed above holds for the three-way competition as well. Due to the doubling time advantage of $T^-$ and homogeneous limitations across cell types, testosterone limitations on $T^p$ and $T^+$ have a higher influence on maintaining coexistence between the cells. In this context, it has been possible to identify zones of resource limitation for both testosterone and oxygen where the system goes from coexistence to $T^-$ domination. Strong testosterone limitation without a numerical advantage of higher initial seeding density for $T^p$ leads to $T^-$ domination with no influence from oxygen. Weak testosterone limitation with a numerical advantage for $T^p$ leads to coexistence with no influence of oxygen. Meanwhile, in the other cases where testosterone limitation is intermediate, the outcomes of competition are pushed either towards coexistence or $T^-$ dominance by the oxygen limitation.

This framework used for studying competition alone then helps us understand the outcomes of therapy in mechanistic terms. $T^p$ and $T^+$ are the only cell types to depend on testosterone and hence only they respond to abiraterone. SOC creates an additional limitation of testosterone by reducing the production rates and hence pushes these two cell types to extinction. AT would have no influence where $T^p$ and $T^+$ are pushed to extinction by competition alone and maximum influence when the tumour is dominated by $T^p$ and $T^+$. The resource limitations and seeding proportions therefore have an influence on the success of AT. The success of AT also depends crucially on the therapy window. Higher window would lead to better success, but at the increased physiological cost of maintaining a larger tumour. Meanwhile, with a smaller window achieving control would be more difficult and there would be a higher risk of competitive release. While this mechanistic information makes for a thorough understanding of the system from first principles, it also highlights the gap between the modelling approach and clinical reality. It is clear that application of such ecologically-aware treatment strategies would also require a quantum shift in the nature of information that can be realistically obtained from a cancer patient over meaningful timescales.

Insofar as eliminating the treatment-resistant cell type, all the therapy strategies we tried have been a failure. But, some of this failure has been informative, and it has been possible to avoid or delay competitive release in some cases by maintaining a non-zero population of the responsive cell types $T^+$ and $T^p$. This is broadly in line with current thinking in the field regarding the goals of AT, which are more focused on achieving control than a complete cure. It is worth noting however, that the total tumour burden even when control was achieved was very close to the model’s maximum effective carrying capacity. This could point to an important gap in the conceptualisation of AT that does not include the physiological cost of control over cure. It may then be worth investigating if AT could also be designed to address ways of minimising this cost alongside tumour control.

This study has been an attempt at a proof of concept to illustrate how ecological dynamics within a cancer system can inform progression as well as therapeutic decisions and outcomes. Based on its results so far, the following lines of further development are worth highlighting:
\begin{enumerate}
  \item The exploration of combination therapy in this study has been limited. While the effects of docetaxel in the model were determined based on available experimental data, the ways in which it can be applied are still open modelling questions that can include the frequency of docetaxel, the magnitude of the dose, and the phase shift between abiraterone and docetaxel. Methods of combination therapy other than docetaxel are also known clinically (radiation, steroids, etc) and some of these can be incorporated into the framework of the current model to understand the dynamics of combination therapy better.
  \item The cell types considered here are an oversimplification of actual cells in a biological system, which are highly variable in their functions and phenotypes. Cancer systems in particular can show an even higher degree of such heterogeneity due to their higher mutation rates and genomic instability. Bringing this to bear on the modelling approach we have taken, exploring a heterogeneity of cellular response across cell types to resource availability would be of interest here due to its strong influence on the outcomes of somatic competition and its effect on therapy. This is how pairwise competition has been explored in this study, and its extension to three-way competition should be informative.
  \item In addition to heterogeneity, the cells can also switch their phenotype based on environmental conditions due to phenotypic plasticity. While some theoretical studies have explored the effect of mutational changes within the context of cell competition \cite{Snippert}, phenotypic plasticity remains understudied. It is possible, however, that implementing plasticity or heterogeneous responses would be much easier with an individual based model, which opens up whole new possibilities of exploring the ecological dynamics of the system.
  \item An individual-based model also allows for the addition of spatial heterogeneity to the system, which is an important component of biological variation in cancer populations. Solid tumours in particular have well-known gradients of resources between the tumour core and edge \cite{Fontaine}, which could again open up even more channels of investigation.
\end{enumerate}

These are possible ways in which the current study could be extended and broadened in scope. However, a more general comment on the modelling approach itself is also justified at this point.

Mathematical modelling is a really powerful tool in biology which helps in understanding a system without use of an actual system for the main experimentation. This can be very useful when experimenting on the actual system or biological model is not possible due to ethics, health risks, etc. However, one must keep in mind that all biological models are simplifications and involve many assumptions. Quoting \cite{Box} ``All models are wrong, but some are useful". There are no objective criteria for what makes a good model, or a fair set of prior assumptions. Every model makes assumptions that are manifestations of constraints inherent to the mathematical framework and data availability.

The concept of carrying capacity in a logistic or Lotka-Volterra system is a debated topic and even more so in cancer systems \cite{McLeod,Deisboeck}. Our model also assumes a carrying capacity derived from a rather arbitrary equilibrium value, $y_i^*$, given in \autoref{K_eq}. However, this arbitrariness is partially alleviated since this carrying capacity is not explicitly fixed and instead is dynamically affected by the current concentration of resources. Nevertheless, it still suffers from the same setbacks of any model that invokes a carrying capacity, which is both difficult to define biologically (especially for cancer populations characterised by ``uncontrolled" growth) and usually impossible to separate from the intrinsic growth rate. This leads to the possibility that a modelling approach that does not involve the carrying capacity at all where the resource availability is directly linked to growth rate in a consumer-resource type model may be better suited to study cancer systems that are typically at the edge of resource availability. Furthermore, comparing the differences between such alternate modelling strategies could itself form an informative study of modelling assumptions and their impact on inferences.

The inherently exponential nature of the model made it very sensitive to parameter values and to small fluctuations in environmental conditions. This is borne out particularly clearly in how this system responds to the application of therapy almost instantaneously. The rapidity of these responses could likely hide subtler dynamics that may be more accessible to a different modelling approach that is designed to pick up on them.

Our modelling approach has been more mechanistic in nature than data driven. We have started with parameters from fundamental processes governing the system for the overall behaviour to emerge out of it. A data driven model on the other hand, fits the parameters to clinical data. One could argue that such a data driven model reflects closer to reality since it follows the same dynamics, however, such models don’t give valuable insight into these fundamental processes and act as black boxes. It may then be useful to explore ways of integrating clinical data more closely into mechanistic models, potentially resulting in mechanistic insight that can be applied more directly to clinical practice.
