\documentclass[11pt,a4paper]{article}
\usepackage{graphicx}
%\usepackage{subcaption}
\usepackage[margin=2cm]{geometry}
\usepackage[maxbibnames=9,citestyle=nature]{biblatex}
\usepackage{hyperref}
\hypersetup{
    colorlinks=true,
    linkcolor=blue,
    filecolor=magenta,
    urlcolor=cyan,
}
\usepackage{mathtools}
\usepackage{bm}
\usepackage{caption}
%\usepackage{multirow}
\DeclareCaptionLabelFormat{cont}{#1~#2\alph{ContinuedFloat}}
\captionsetup[ContinuedFloat]{labelformat=cont}
\newcommand{\HRule}{\rule{\linewidth}{0.5mm}}

\title{\textsc{Studying the effects of competition on adaptive therapy}\\\Large{Mid Year Report}}
\author{Harshavardhan BV (20161100), 5th Year BS-MS\\
Under the guidance of Prof. Sutirth Dey, IISER Pune}
\date{January 2021}
\bibliography{References}
\graphicspath{ {./images/} }

\begin{document}
\maketitle

\section{Introduction}
\subsection{Adaptive therapy}
Conventional therapy against cancer focusses on minimising tumour burden by administering cytotoxic drugs at the maximum tolerated dosage (MTD). However, most tumours are heterogenous in their sensitivity to these drugs, and such MTD regimens eliminate the most sensitive cells, allowing the resistant cells to re-establish a resistant population \cite{Scott}. Adaptive therapy aims to reduce such competitive release by administering the drug at a dosage less than the MTD in a fluctuating manner. This allows for some proportion of sensitive cells to survive in the tumour population while reducing the net tumour burden, thus preventing a complete takeover by the resistant phenotype.

\subsection{Metastatic Castration-Resistant Prostate Cancer}
The system of study was chosen to be Metastatic Castration-Resistant Prostate Cancer (mCRPC) since it has a history of adaptive therapy theory work \cite{Cunningham}. mCRPC is composed of three types of cells: $T^+$, $T^p$ and $T^-$. Only $T^+$ and $T^p$ require testosterone for proliferation and survival, and $T^-$ is testosterone-independent. While $T^+$ cannot secrete testosterone, $T^p$ cells produce testosterone through upregulation of the corresponding enzyme, $CYP17\alpha$ and $T^-$ cells have mutations in androgen receptors that allow them to survive without testosterone.

\subsection{Competition between cells}
The success of adaptive therapy in containing the tumour depends on the effectiveness of competition between sensitive and resistant cells. Cells can use different strategies such as higher proliferation rate, better survival at sub-optimal conditions or lower death rate to compete with each other, and several such strategies are seen to be acquired over the course of cancer progression, as shown by the ``hallmarks of cancer" framework \cite{Hanahan}. The role of the choice of strategy on the final outcomes of cell competition and eventually, the efficacy of adaptive therapy, has not yet been studied closely.

\section{Work done till now}
\subsection{ODE Model}
To begin with, a simplistic Ordinary Differential Equation (ODE) model was used. This would help us with forming expectations of the system and parameterization for the upcoming Agent Based Model (ABM). The ODE model is less computationally costly at the tradeoff of not being able to capture complex behaviour when compared to ABM.

The model we developed is based on a simple Logistic Model \cite{Logistic} but with an addition environmental dependence on carrying capacity. The ``environment" consists of the resources, oxygen and testosterone which have their own equations for production and consumption. Although a real cell would depend on a lot more resources, for simplicity they're all assumed to be in excess. The equations are given below:
\begin{equation}
  \frac{dy_i}{dt} = r_i y_i (1 - \frac{y_i}{K_{min} + \rho_i f_i(O_2) f_i(test)} )- \delta_i y_i
  \label{celleq}
\end{equation}
\begin{equation}
  \frac{dO_2}{dt} = p_{O_2} - \sum_i \mu_{O_2,i} y_i - \lambda_{O_2} O_2
  \label{o2eq}
\end{equation}
\begin{equation}
  \frac{dtest}{dt} = p_{test} y_{T^p} - \sum_i \mu_{test,i} y_i - \lambda_{test} test
  \label{testeq}
\end{equation}
\begin{equation}
  f_i(res) = \begin{cases}
    1 &\text{if } ul_{res,i} \leq res\\
    \frac{res-ll_{res,i}}{ul_{res,i}-ll_{res,i}} &\text{if } ll_{res,i} < res < ul_{res,i}\\
    0 &\text{if } res \leq ll_{res,i}\\
  \end{cases}
  \label{freseq}
\end{equation}

$i \in \{T^+,T^p,T^-\}$ and $res \in \{O_2,test\}$.

\subsection{Parameters and Standardization}
A major portion of the time was spent in standardising the model with suitable parameters. The following Table \ref{parmtable} gives a brief description of the parameters, the values used, and the source for the same. Note that all the resource parameters are normalised to tissue levels of that resource.\\
Some of the constraints are derived from the following equations.
\begin{equation}
  r_i = \frac{ln(2)}{\tau_{d,i}} + \delta_i
  \label{r_eq}
\end{equation}
\begin{equation}
  \rho_i=\frac{r_i}{r_i-\delta_i} y_i^*
  \label{rho_eq}
\end{equation}
\begin{equation}
  p_{O_2} = \lambda_{O_2} O_2^* + y_i^* \mu_i
  \label{p_o2_eq}
\end{equation}
\begin{equation}
  p_{test} - \mu_{test,T^p} = \frac{test^* \lambda_{test}}{y_{T^p}^*} = 4 \times 10^{-4}
  \label{p_test_eq}
\end{equation}
\begin{table}[t]
  \centering
  \begin{tabular}{|l|p{5cm}|c|l|}
    \hline
    \textbf{Parameter}  & \textbf{Description} & \textbf{Value(s)} & \textbf{Source(s)} \\ \hline
    $y_i$ & No. of cells of cell type $i$ & N/A & N/A  \\ \hline
    $r_i$ & Population growth rate of cell type i  &
    \begin{tabular}{l|l}
      $T^+$ & $2.84 \times 10^{-3}$ min$^{-1}$\\
      $T^p$ & $2.79 \times 10^{-3}$ min$^{-1}$\\
      $T^-$ & $6.23 \times 10^{-4}$ min$^{-1}$\\
    \end{tabular}
    & Eq \ref{r_eq} \\ \hline
    $\delta_i$  & Population death rate of cell type i &
    \begin{tabular}{l|l}
      $T^+$ & $2.5 \times 10^{-3}$ min$^{-1}$\\
      $T^p$ & $2.5 \times 10^{-3}$ min$^{-1}$\\
      $T^-$ & $1.6 \times 10^{-4}$ min$^{-1}$\\
    \end{tabular}
    & \cite{Jain}  \\ \hline
    $K_{min}$ & Carrying capacity in the absence of resources & 1 & essential  \\ \hline
    $\rho_i$ & Carrying capacity coming up through the environment/resources &
    \begin{tabular}{l|l}
      $T^+$ & $8.35 \times 10^4$ \\
      $T^p$ & $9.62 \times 10^4$ \\
      $T^-$ & $1.34 \times 10^4$ \\
    \end{tabular}
    & Eq \ref{rho_eq} \\ \hline
    $f_{i,res}$ & Functional dependence of cell type $i$ on resource $res$, normalised to 1 & $f_{T^-,test}=1$ & N/A \\ \hline
    $p_{res}$ & Production rate of resource, either as bulk or by cells &
    \begin{tabular}{l|l}
      $O_2$ & 0.11 min$^{-1}$\\
      $test$ & $5 \times 10^{-7}$ min$^{-1}$cell$^{-1}$\\
    \end{tabular}
    & Eqs \ref{p_o2_eq},\ref{p_test_eq}\\ \hline
    $\mu_{res,i}$ & Uptake of resource $res$ by cell type $i$ &
    \begin{tabular}{l|l|l}
      $O_2$ & $T^+$ & $1.63 \times 10^{-6}$ min$^{-1}$cell$^{-1}$\\
      & $T^p$ & $1.63 \times 10^{-6}$ min$^{-1}$cell$^{-1}$\\
      & $T^-$ & $1.04 \times 10^{-6}$ min$^{-1}$cell$^{-1}$\\ \hline
      $test$ & $T^+$ & $2.34 \times 10^{-8}$ min$^{-1}$cell$^{-1}$\\
      & $T^p$ & $6.00 \times 10^{-8}$ min$^{-1}$cell$^{-1}$\\
      & $T^-$ & 0 min$^{-1}$cell$^{-1}$\\
    \end{tabular}
    & \cite{HailJr}, Eq \ref{p_test_eq}\\ \hline
    $\lambda_{res}$ & Decay rate of resource $res$ &
    \begin{tabular}{l|l}
      $O_2$ & 0.100 min$^{-1}$\\
      $test$ & 0.004 min$^{-1}$\\
    \end{tabular}
    & \cite{Jain}\\ \hline
    $ll_{res,i}$ & Lower limit/threshold level of resource $res$ for carrying capacity of cell type $i$ & $\in [0,1]$ & N/A \\ \hline
    $ul_{res,i}$ & Upper limit/saturation level of resource $res$ for carrying capacity of cell type $i$ & $\in [0,1]$ & N/A \\ \hline
    \multicolumn{4}{|c|}{Supplementary Parameters}\\ \hline
    $\tau_d$  & Doubling time of cell type $i$ &
    \begin{tabular}{l|l}
      $T^+$ & $34$ hr \\
      $T^p$ & $40$ hr \\
      $T^-$ & $25$ hr \\
    \end{tabular}
    & \cite{atcc} \\ \hline
    $y_i^*$ & Equilibrium value of cell number in absence of competition & 10000 & assumed \\ \hline
    $res^*$ & Equilibrium/Tissue levels of resource with one cell type present &
    \begin{tabular}{l|l}
      $O_2$    & 2.5 mmHg          \\
      $test$   & 3.74 pmol/g tissue\\
    \end{tabular}
    & \cite{Steward},\cite{Titus} \\ \hline

  \end{tabular}
  \caption{Table of all parameters}
  \label{parmtable}
\end{table}

\subsection{Pairwise Competition}
With the standardized model, runs with $T^p - T^-$, and $T^+ - T^p$ were done. Here, the parameters of limits were varied as combinations of:
\begin{enumerate}
  \item Varying $ul_{res,i}$ with other parameters fixed
  \item Varying $ll_{res,i}$ with other parameters fixed
  \item Changing $ll_{res,i}$ and $ul_{res,i}$ by the same amount
  \item Changing $\mu_{test,T+}$ for $T^+ - T^p$ pair
\end{enumerate}
From the runs, the following observations were made:
\begin{itemize}
  \item In the $T^p - T^-$ pair:
  \begin{enumerate}
    \item $T^p$ goes extinct most of the time. This is because $T^p$ is highly limited by testosterone and $T^-$ manages to outcompete it as it only depends on oxygen.
    \item Even when $T^-$ is highly oxygen limited (high $ll_{O_2,T^-}$), there is a initial dip but $T^-$ wins out as oxygen levels rise faster than testosterone.
    \item Only at low $ul_{test,T^p}$ can $T^p$ be sustainable or outcompete $T^-$ as low levels of testosterone is enough to give enough growth in this case. $T^p$ becomes oxygen limited and competes with $T^-$ for oxygen.
    \item The outcomes of which cell type comes out on top depends on the initial condition as well. The
  \end{enumerate}
  \item In the $T^+ - T^p$ pair:
  \begin{enumerate}
    \item Due to the severe testosterone and density dependence, $T^+$ can pull down $T^p$ levels and both go extinct as a result, most of the time.
    \item When $T^+$ is highly oxygen limited (high $ll_{O_2,T^+}$), $T^p$ is able to grow initially and secrete enough testosterone in the environment to sustain a small portion of $T^+$ if they're not extinct already.
    \item There is coexistence of both cell types when $ul_{test,T^+} \geq ul_{test,T^p}$. By this, $T^p$ is able to grow faster initially to provide enough testosterone to support $T^+$.
    \item As $ul_{test,T^+}$ increases in the above case, $T^p$ proportion increases in the final population.
  \end{enumerate}
\end{itemize}

\section{Future Plans}
For the pairwise runs, more combinations of parameters have to be explored. $O_2$ limits have to be explored with low $ul_{test,T^p}$ to have oxygen limited case for $T^p$. A situation where $p_{O_2}$ is lower could as be tried to have oxygen limitations. Different initial conditions have to be tried to check if the outcome changes on doing the same.

After this, we plan to move onto competitive runs between all three cell types and then towards simulating different regimens of adaptive therapy. Therapy in this model would involve a $p_{test} = f(dose)$.

Parallely an ABM would be developed which has spatially explicity positioning of cells and diffusion of resources. The simulations run with ODE model would be replicated in the ABM and a comparison between the outcomes would be made.

\printbibliography
\end{document}
