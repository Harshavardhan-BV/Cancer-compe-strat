\documentclass[11pt,a4paper]{article}
\usepackage{graphicx}
%\usepackage{subcaption}
\usepackage[margin=2cm]{geometry}
\usepackage[maxbibnames=9,citestyle=nature]{biblatex}
\usepackage{hyperref}
\hypersetup{
    colorlinks=true,
    linkcolor=blue,
    filecolor=magenta,
    urlcolor=cyan,
}
\usepackage{mathtools}
\usepackage{bm}
\usepackage{caption}
%\usepackage{multirow}
\DeclareCaptionLabelFormat{cont}{#1~#2\alph{ContinuedFloat}}
\captionsetup[ContinuedFloat]{labelformat=cont}
\newcommand{\HRule}{\rule{\linewidth}{0.5mm}}

\title{\textsc{Studying the effects of competition on adaptive therapy}\\\Large{Mid Year Report}}
\author{Harshavardhan BV (20161100), 5th Year BS-MS\\
Under the guidance of Prof. Sutirth Dey, IISER Pune}
\date{January 2021}
\bibliography{References}
\graphicspath{ {./images/} }

\begin{document}
\maketitle

\section{Introduction}
\subsection{Adaptive therapy}
Conventional therapy against cancer focusses on minimising tumour burden by administering cytotoxic drugs at the maximum tolerated dosage (MTD). However, most tumours are heterogenous in their sensitivity to these drugs, and such MTD regimens eliminate the most sensitive cells, allowing the resistant cells to re-establish a resistant population \cite{Scott}. Adaptive therapy aims to reduce such competitive release by administering the drug at a dosage less than the MTD in a fluctuating manner. This allows for some proportion of sensitive cells to survive in the tumour population while reducing the net tumour burden, thus preventing a complete takeover by the resistant phenotype.

\subsection{Metastatic Castration-Resistant Prostate Cancer}
The system of study was chosen to be Metastatic Castration-Resistant Prostate Cancer (mCRPC) since it has a history of adaptive therapy theory work \cite{Cunningham}. mCRPC is composed of three types of cells: $T^+$, $T^p$ and $T^-$. Only $T^+$ and $T^p$ require testosterone for proliferation and survival, and $T^-$ is testosterone-independent. While $T^+$ cannot secrete testosterone, $T^p$ cells produce testosterone through upregulation of the corresponding enzyme, $CYP17\alpha$ and $T^-$ cells have mutations in androgen receptors that allow them to survive without testosterone.

\subsection{Competition between cells}
The success of adaptive therapy in containing the tumour depends on the effectiveness of competition between sensitive and resistant cells. Cells can use different strategies such as higher proliferation rate, better survival at sub-optimal conditions or lower death rate to compete with each other, and several such strategies are seen to be acquired over the course of cancer progression, as shown by the ``hallmarks of cancer" framework \cite{Hanahan}. The role of the choice of strategy on the final outcomes of cell competition and eventually, the efficacy of adaptive therapy, has not yet been studied closely.

\section{Work done till now}
\subsection{ODE Model}
To begin with, a simplistic Ordinary Differential Equation (ODE) model was used. This would help us with forming expectations of the system and parameterization for the upcoming Agent Based Model (ABM). The ODE model is less computationally costly at the tradeoff of not being able to capture complex behaviour when compared to ABM.

The model we developed is based on a simple Logistic Model \cite{Logistic} but with an addition environmental dependence on carrying capacity. The ``environment" consists of the resources, oxygen and testosterone which have their own equations for production and consumption. Although a real cell would depend on a lot more resources, for simplicity they're all assumed to be in excess. The equations are given below:
\begin{equation}
  \frac{dy_i}{dt} = r_i y_i (1 - \frac{\sum_j y_j}{1 + K_{i,max} f_i(O_2) f_i(test)} )- \delta_i y_i
  \label{celleq}
\end{equation}
\begin{equation}
  \frac{dO_2}{dt} = p_{O_2} - \sum_i \mu_{O_2,i} y_i - \lambda_{O_2} O_2
  \label{o2eq}
\end{equation}
\begin{equation}
  \frac{dtest}{dt} = p_{test} y_{T^p} - \sum_i \mu_{test,i} y_i - \lambda_{test} test
  \label{testeq}
\end{equation}
\begin{equation}
  f_i(res) = \begin{cases}
    1 &\text{if } ul_{res,i} \leq res\\
    \frac{res-ll_{res,i}}{ul_{res,i}-ll_{res,i}} &\text{if } ll_{res,i} < res < ul_{res,i}\\
    0 &\text{if } res \leq ll_{res,i}\\
  \end{cases}
  \label{freseq}
\end{equation}

$i \in \{T^+,T^p,T^-\}$ and $res \in \{O_2,test\}$.

\subsection{Parameters and Standardization}
A major portion of the time was spent in standardising the model with suitable parameters. The following Table \ref{parmtable} gives a brief description of the parameters, the values used, and the source for the same. Note that all the resource parameters are normalised to tissue levels of that resource.\\
Some of the constraints are derived from the following equations.
\begin{equation}
  r_i = \frac{ln(2)}{\tau_{d,i}} + \delta_i
  \label{r_eq}
\end{equation}
\begin{equation}
  K_{i,max}=\frac{r_i}{r_i-\delta_i} y_i^*
  \label{rho_eq}
\end{equation}
\begin{equation}
  p_{O_2} = \lambda_{O_2} O_2^* + y_i^* \mu_i
  \label{p_o2_eq}
\end{equation}
\begin{equation}
  p_{test} - \mu_{test,T^p} = \frac{test^* \lambda_{test}}{y_{T^p}^*} = 4 \times 10^{-4}
  \label{p_test_eq}
\end{equation}
\begin{table}[t]
  \centering
  \begin{tabular}{|l|p{5cm}|c|l|}
    \hline
    \textbf{Parameter}  & \textbf{Description} & \textbf{Value(s)} & \textbf{Source(s)} \\ \hline
    $y_i$ & No. of cells of cell type $i$ & N/A & N/A  \\ \hline
    $r_i$ & Population growth rate of cell type i  &
    \begin{tabular}{l|l}
      $T^+$ & $2.84 \times 10^{-3}$ min$^{-1}$\\
      $T^p$ & $2.79 \times 10^{-3}$ min$^{-1}$\\
      $T^-$ & $6.23 \times 10^{-4}$ min$^{-1}$\\
    \end{tabular}
    & Eq \ref{r_eq} \\ \hline
    $\delta_i$  & Population death rate of cell type i &
    \begin{tabular}{l|l}
      $T^+$ & $2.5 \times 10^{-3}$ min$^{-1}$\\
      $T^p$ & $2.5 \times 10^{-3}$ min$^{-1}$\\
      $T^-$ & $1.6 \times 10^{-4}$ min$^{-1}$\\
    \end{tabular}
    & \cite{Jain}  \\ \hline
    $K_{i,max}$ & Maximum Carrying capacity, coming up through the environment/resources &
    \begin{tabular}{l|l}
      $T^+$ & $8.35 \times 10^4$ \\
      $T^p$ & $9.62 \times 10^4$ \\
      $T^-$ & $1.34 \times 10^4$ \\
    \end{tabular}
    & Eq \ref{rho_eq} \\ \hline
    $f_{i,res}$ & Functional dependence of cell type $i$ on resource $res$, normalised to 1 & $f_{T^-,test}=1$ & N/A \\ \hline
    $p_{res}$ & Production rate of resource, either as bulk or by cells &
    \begin{tabular}{l|l}
      $O_2$ & 0.11 min$^{-1}$\\
      $test$ & $5 \times 10^{-7}$ min$^{-1}$cell$^{-1}$\\
    \end{tabular}
    & Eqs \ref{p_o2_eq},\ref{p_test_eq}\\ \hline
    $\mu_{res,i}$ & Uptake of resource $res$ by cell type $i$ &
    \begin{tabular}{l|l|l}
      $O_2$ & $T^+$ & $1.63 \times 10^{-6}$ min$^{-1}$cell$^{-1}$\\
      & $T^p$ & $1.63 \times 10^{-6}$ min$^{-1}$cell$^{-1}$\\
      & $T^-$ & $1.04 \times 10^{-6}$ min$^{-1}$cell$^{-1}$\\ \hline
      $test$ & $T^+$ & $2.34 \times 10^{-8}$ min$^{-1}$cell$^{-1}$\\
      & $T^p$ & $6.00 \times 10^{-8}$ min$^{-1}$cell$^{-1}$\\
      & $T^-$ & 0 min$^{-1}$cell$^{-1}$\\
    \end{tabular}
    & \cite{HailJr}, Eq \ref{p_test_eq}\\ \hline
    $\lambda_{res}$ & Decay rate of resource $res$ &
    \begin{tabular}{l|l}
      $O_2$ & 0.100 min$^{-1}$\\
      $test$ & 0.004 min$^{-1}$\\
    \end{tabular}
    & \cite{Jain}\\ \hline
    $ll_{res,i}$ & Lower limit/threshold level of resource $res$ for carrying capacity of cell type $i$ & $\in [0,1]$ & N/A \\ \hline
    $ul_{res,i}$ & Upper limit/saturation level of resource $res$ for carrying capacity of cell type $i$ & $\in [0,1]$ & N/A \\ \hline
    \multicolumn{4}{|c|}{Supplementary Parameters}\\ \hline
    $\tau_d$  & Doubling time of cell type $i$ &
    \begin{tabular}{l|l}
      $T^+$ & $34$ hr \\
      $T^p$ & $40$ hr \\
      $T^-$ & $25$ hr \\
    \end{tabular}
    & \cite{atcc} \\ \hline
    $y_i^*$ & Equilibrium value of cell number in absence of competition & 10000 & assumed \\ \hline
    $res^*$ & Equilibrium/Tissue levels of resource with one cell type present &
    \begin{tabular}{l|l}
      $O_2$    & 2.5 mmHg          \\
      $test$   & 3.74 pmol/g tissue\\
    \end{tabular}
    & \cite{Steward},\cite{Titus} \\ \hline

  \end{tabular}
  \caption{Table of all parameters}
  \label{parmtable}
\end{table}

\subsection{Pairwise Competition}
With the standardized model, runs with $T^p - T^-$, and $T^+ - T^p$ were done over some combinations of parameters. From the runs, the following observations were made:
\begin{itemize}
  \item In the $T^p - T^-$ pair:
  \begin{enumerate}
    \item $T^p$ is limited by both testosterone and oxygen, whereas, $T^-$ is only limited by oxygen. The testosterone limitation is contolled through $ll_{test,T^p}$ and $ul_{test,T^p}$.
    \item Only when $T^p$ is not severely testosterone limited, through lower $ul_{test,T^p}$, can $T^p$ coexist with or outcompete $T^-$. In every other case, $T^-$ drives $T^p$ to extinction.
    \item  Even when $T^-$ is highly oxygen limited, through high $ll_{O2,T-}$, and $T^p$ limited by testosterone, $T^-$ wins out eventually as oxygen levels rise faster than testosterone.
    \item The outcomes on which cell comes out on top also depends on the initial proportion of $T^p$ for the other parameters being the same.
  \end{enumerate}
  \item In the $T^+ - T^p$ pair:
  \begin{enumerate}
    \item Both $T^+$ and $T^p$ are limited by both oxygen and testosterone and compete for these resources.
    \item When both are severly limited by testosterone, $T^+$ can consume and grow through the limited testosterone present and this is enough for the density dependent competition to drive $T^p$ to extinction. Without $T^p$ to provide testosterone $T^+$ also goes extinct.
    \item When $T^+$ is severly limited by oxygen and $T^p$ not, $T^p$ is able to utilise the initial period to grow and secrete testosterone in the environment to sustain a small $T^+$ population if they are not extinct.
    \item When $T^p$ is limited less by testosterone than $T^+$, through $ul_{test,Tp} \leq ul_{test,T+}$, can both cells coexist. Through the lower testosterone limitation, $T^p$ can grow faster initially and secrete enough testosterone for $T^+$ without being negatively affected by $T^+$.
    \item In the above case, the proportion of $T^+$ in the final population decreases as $T^+$ becomes more testosterone limited.
  \end{enumerate}
\end{itemize}

\section{Future Plans}
From the observations, it is clear that testosterone is limiting a lot. More runs have to be done where the testosterone limitation is relaxed. The oxygen limit needs exploration with lower $ul_{test,i}$ than currently done. Alternatively, oxygen can be made more limiting than testosterone by changing the production rates of these resources.

After this, we plan to move onto competitive runs between all three cell types and then towards simulating different regimens of adaptive therapy. Therapy in this model would involve a $p_{test} = f(dose)$.

Parallely an ABM would be developed which has spatially explicity positioning of cells and diffusion of resources. The simulations run with ODE model would be replicated in the ABM and a comparison between the outcomes would be made.

\printbibliography
\end{document}
